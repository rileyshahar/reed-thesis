% vim:ft=tex

\section{Categories}

% TODO: introduction and motivation of some sort

\begin{dfn}[Category]\label{def:category}
A \emph{category} $\cat{C}$ consists of the following data:
\begin{itemize}
  \item a collection\footnotemark{} of objects, overloadingly also called $\cat{C}$;
	\item for each pair of objects $x,y \in  \cat{C}$, a collection of \emph{morphisms} $\cat{C}(x, y)$;
	\item for each object $x \in \cat{C}$, a designated \emph{identity morphism} $x \xrightarrow{1_x}  x$;
  \item for each pair of morphisms $x \xrightarrow{f}  y \xrightarrow{g}  z$, a designated \emph{composite morphism} $x \xrightarrow{gf}  z$.
\end{itemize}
This data must satisfy the following axioms:
\begin{itemize}
  \item \emph{unitality}: for any $x \xrightarrow{f}  y$, $1_yf = f = f1_x$;
  \item \emph{associativity}: for any $x \xrightarrow{f} 		y \xrightarrow{g} z \xrightarrow{h} w$, $(hg)f = h(gf)$.
\end{itemize}
\end{dfn}

\footnotetext{We use the word \emph{collection} for foundational reasons: in
many important examples, the objects and morphisms do not form sets. We ignore
such foundational issues here; they are discussed in~\cite[Section
1.6]{maclane-1971}.}

\noindent
Categories are widespread in mathematics, as the following examples show.

\begin{ex}[Concrete Categories]\label{ex:concrete categories}The following are all categories:
  \begin{itemize}
    \item $\scat{Set}$ is the category of sets and functions.
    \item $\scat{Grp}$ is the category of groups and group homomorphisms.
    \item $\scat{Ring}$ is the category of rings and ring homomorphisms.
    \item $\scat{Top}$ is the category of topological spaces and homeomorphisms.
    \item For any field $\kk$, $\scat{Vect}_\kk$ is the category of vector
      spaces over $\kk$ and linear transformations.
  \end{itemize}
\end{ex}

\noindent
We call such categories, whose objects are structured sets and whose morphisms
are structure-preserving set-functions, \emph{concrete}. On the other hand, many
categories look quite different.

\begin{ex}\label{ex:abstract categories}The following are also categories:
  \begin{itemize}
    \item The \emph{empty category} has no objects and no morphisms.
    \item The \emph{trivial category} has a single object and its identity morphism.
    \item Any group (or, more generally, monoid) can be thought of as a category
      with a single object, a morphism for every element, and composition
      given by the monoid multiplication.
    \item Any poset (or, more generally, preorder) $(P, \leq)$ can be thought
      of as a category whose objects are the elements of $P$, with a unique
      morphism $x\rightarrow y$ if and only if $x\leq y$. In this sense,
      composition is a ``higher-dimensional'' transitivity, and identities are
      higher-dimensional reflexivity.
    \item Associated to any directed graph is the \emph{free category} on the
      graph, whose objects are nodes and whose morphisms are paths.
      % This higher dimensional stuff is cool and it's how I think about these
      % objects, but probably not necessary for our purposes. --riley
		\item There is a category whose objects are (roughly) multisets of molecules
			and whose morphisms are chemical reactions. See \cite{baez-2017} for a
			formalization of this notion.
  \end{itemize}
\end{ex}

% \begin{ntn}
%   There are many common notational conventions for categories. For clarity, we
%   will sometimes write $g\circ f$ for the composite $gf$.
% \end{ntn}

\noindent
When working with categories, we often want to show that two complex composites
equate. In this case, we prefer graphical notation to the more traditional
symbolic equalities of \Cref{def:category}. The key idea is that such diagrams
can be ``pasted'', allowing us to build up complex equalities from simpler ones.

\begin{dfn}[Commutative Diagram]\label{def:commutative diagram}
  A diagram \emph{commutes} if, for any pair of paths through the
  diagram with the same start and end, the composite morphisms are equal.
\end{dfn}

\noindent
The notion of a diagram can be made precise fairly easily; see~\cite[Section 1.6]{riehl-2017}.

\begin{ex}
In this language, the axioms of \Cref{def:category} are expressed by commutativity of the
following diagrams:
\begin{figure}[H]
  \centering
  \begin{tikzcd}
    x\ar[r, "f"]\ar[rr, "gf", bend left=60] &
    y\ar[r, "g"]\ar[rr, "hg"', bend right=60] &
    z\ar[r, "h"] & w
  \end{tikzcd}
  \begin{tikzcd}
    x\ar[r, "1_x"]\ar[rd, "f"'] & x\ar[d, "f"] \\
    & y
  \end{tikzcd}
  \begin{tikzcd}
    x\ar[r, "f"]\ar[rd, "f"'] & y\ar[d, "1_y"] \\
    & y\punctuation{.}
  \end{tikzcd}
\end{figure}
\end{ex}
