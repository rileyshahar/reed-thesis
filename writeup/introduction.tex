

Cryptography has a composition problem: category theory is a mathematical theory
of composition.

In some sense, this thesis has three goals: to give cryptographers the necessary
background in category theory to evaluate categorical frameworks for themselves, to
give category theorists the enough background in cryptography for them to
motivate their work towards the potential applications, and to summarize the
progress and open questions that exist. The thesis is divided along those lines.

In \Cref{chap:cryptography}, we give an introduction
to the foundations of cryptography, focusing on definitions and examples
relevant to the study of composability. Cryptographers can safely skip this
chapter, though cryptographers new to questions of composability may be
interested in \Cref{sec:crypto-composition}, which
presents several of the central difficulties.

In \Cref{chap:category-theory}, we give an
introduction to category theory oriented towards computer scientists. There are
many excellent books with this aim; ours distinguishes itself through its focus
on monoidal categories, coherence axioms, and in particular string diagrams,
which form a powerful graphical language for reasoning about computational
objects. As in the previous chapter, the aim is to get to the necessary
background for cryptography as quickly as possible, and so we skip several
standard topics of substantial interest to computer scientists. Category
theorists who are comfortable working with string diagrams can safely skip this
chapter, though there are several examples of computational applications that
may be of interest.
