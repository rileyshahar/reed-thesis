To first approximation, cryptography is the \emph{mathematical study of secure
computation}. In a computation, we want to use \emph{protocols} to transform
\emph{resources}. For a computation to be secure, it must successfully resist
\emph{attacks} by \emph{adversaries}. This is an extremely broad scope:
cryptography includes secure communication, private data analysis,
password-based authentication, distributed consensus-making, fault-tolerance of
sensor systems, and many other applications.

In the modern world, computational systems do not run on their own. You may be
securely communicating with your bank on one tab, have email open on another
tab, and be syncing photos from your phone in the background. After that
communication, your bank may want to analyze your data in a way that respects
your privacy. All this happens for millions of people simultaneously. As such,
we cannot study cryptographic protocols in a vacuum: we need to consider how
they behave in concert with other computational systems. This is the
\emph{problem of cryptographic composability}.

As we will see, most frameworks for handling composability, including the
popular Universal Composability~\cite{canetti-2000}, rely on precise low-level
machine models. Proofs in these frameworks are only technically valid if their
protocols can be encoded into the machine model---and if that encoding satisfies
certain technical hypotheses which do not generally hold. This state of affairs
poses significiant issues for both the feasibility of writing proofs in these
frameworks, and---because of the general complexity of the underlying
machine models---for the trustworthiness of those proofs.

A natural way out is to give an axiomatization of the properties such a machine
model, or a theory of cryptography generated from it, should satisfy. If a
composition theorem can be proven for any theory satisfying these axioms, then
proofs would not have to use complex machine models except when that complexity
is necessary to the development of the protocol. We need an algebric model of
computation for this to work: category theory is such a model.

Category theory was first used for cryptographic composability by
\citeauthor{broadbent-karvonen-2022}~\cite{broadbent-karvonen-2022}. While we
give several original contribution in
\Cref{chap:categorical-cryptography},
to first order, this thesis is an exposition and evaluation of their framework.
The key idea is that category theory gives us sensible boundaries of
abstraction: we can separately treat computation, interaction, and security, and
evaluate each such treatment on their own.

In some sense, this thesis has three goals: to give cryptographers the necessary
background in category theory to evaluate categorical frameworks for themselves,
to give category theorists the enough background in cryptography for them to
motivate their work towards the potential applications, and to evaluate the
existing literature and pose some barriers and open questions to a successful
categorical theory of cryptographic composability. The thesis is divided along
those lines.

In \Cref{chap:cryptography}, we give an introduction
to the foundations of cryptography, focusing on definitions and examples
relevant to the study of composability. Cryptographers can safely skip this
chapter, though cryptographers new to questions of composability may be
interested in \Cref{sec:crypto-composition}, which
presents several of the central difficulties.

In \Cref{chap:category-theory}, we give an
introduction to category theory oriented towards computer scientists. There are
many excellent books with this aim; our narrative distinguishes itself through
its focus on monoidal categories, coherence axioms, and in particular string
diagrams, which form a powerful graphical language for reasoning about
computational objects. As in the previous chapter, the aim is to get to the
necessary background for cryptography as quickly as possible, and so we skip
several standard topics of substantial interest to computer scientists. Category
theorists who are comfortable working with string diagrams can safely skip this
chapter, though there are several examples of computational applications that
may be of interest.

In \Cref{chap:categorical-cryptography},
we present the model of~\cite{broadbent-karvonen-2022}. We also present several
original contributions: \begin{itemize}
  \item In
    \Cref{sec:ef-comp},
    we show how to combine binary-encoded sets with Kleisi categories to give
    categorical models of computationally-bounded, effectful computation.
  \item In \Cref{sec:extensions},
    we give several extensions to the semantics of protocols
    of~\cite{broadbent-karvonen-2022}, including tools for simultaneously handling
    single- and multi-use resources, and for modelling protocols with correlated
    input.
  \item In \Cref{sec:interactive proof},
    we use these novel tools to give a categorical correctness definition for
    interactive proofs.
  \item In \Cref{sec:2-cat},
    we generalize the model of~\cite{broadbent-karvonen-2022} to a 2-categorical
    setting.
  \item Finally, in \Cref{sec:conclusion}, we discuss
    several other lines of work, and give a heurstic comparison between the
    categorical model and Universal Composability.
\end{itemize}
