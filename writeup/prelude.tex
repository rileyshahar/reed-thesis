% must-haves
\usepackage{amssymb}
\usepackage{amsmath}
\usepackage{amsthm}
\usepackage{amsfonts}
\usepackage{amsxtra}

\usepackage{todonotes}
\usepackage{multicol}
\usepackage{xfrac}

% fonts
\usepackage{bbm}
\usepackage{graphicx}

\definecolor{cyan}{RGB}{102,204,238}
\definecolor{red}{RGB}{238,102,119}
\definecolor{green}{RGB}{34,136,51}

% links/refs
\usepackage[
	unicode=true,
	pdfusetitle,
	bookmarks=true,
	bookmarksnumbered=false,
	breaklinks=false,
	backref=false,
	colorlinks=true,
	linkcolor=blue,
	citecolor=blue,
	urlcolor=blue,
	final
]{hyperref}
\usepackage{bookmark}
\usepackage[
	backend=biber,
	style=alphabetic
]{biblatex}
\addbibresource{references.bib}
\usepackage[capitalise]{cleveref}

% figs
\usepackage{float}
\usepackage{tikz-cd}
\usepackage{caption}
\usepackage{subcaption}
\usepackage{mathpartir}
\usepackage{adjustbox}
\usetikzlibrary{matrix, arrows, cd, patterns}

% diagrams
../writeup/diagrams.tex

% formatting
\usepackage{enumitem}

% crypto
\usepackage[
	n,
	advantage,
	operators,
	sets,
	adversary,
	landau,
	probability,
	notions,
	logic,
	ff,
	mm,
	primitives,
	events,
	complexity,
	oracles,
	asymptotics,
	keys
]{cryptocode}

% cross-reference format
\creflabelformat{equation}{#2#1#3}
\newlist{questions}{enumerate}{1}
\setlist[questions]{label = \arabic*., ref=\arabic*}

\crefname{questionsi}{question}{questions}
\Crefname{questionsi}{Question}{Questions}

% quality of life
\renewcommand{\emptyset}{\varnothing}
\newcommand{\denote}[1]{\llbracket #1 \rrbracket}

% diagrams
\newcommand{\punctuation}[1]{\makebox[0pt][l]{#1}}

% font conventions
\newcommand{\cat}[1]{{\mathcal{#1}}}
\newcommand{\scat}[1]{{\textsc{\text{#1}}}}
\newcommand{\alg}[1]{{\texttt{#1}}}

\DeclareMathOperator{\cod}{cod}
\DeclareMathOperator{\dom}{dom}
\DeclareMathOperator{\ca}{ca}
\DeclareMathOperator{\view}{view}
\DeclareMathOperator{\exec}{exec}
\DeclareMathOperator{\ideal}{ideal}
\let\state\relax
\newcommand{\state}[1]{\operatorname*{st}(#1)}
\newcommand{\onecomb}[1]{\operatorname*{1-comb}(#1)}
\newcommand{\ncomb}[1]{\operatorname*{n-comb}(#1)}
\newcommand{\ncombb}[1]{\operatorname*{n-comb!}(#1)}
\newcommand{\ncomba}[1]{\operatorname*{n-comb}^*(#1)}
\let\prot\relax
\newcommand{\prot}[2]{\operatorname{prot}_{#1}(#2)}
\newcommand{\prota}[2]{\operatorname{prot}^*_{#1}(#2)}
\newcommand{\protb}[2]{\operatorname{prot}!_{#1}(#2)}
\newcommand{\pred}[1]{\operatorname{pred}(#1)}
\newcommand{\oprot}[2]{\overline{\operatorname{prot}}_{#1}(#2)}
\newcommand{\oprota}[2]{\overline{\operatorname{prot}}^*_{#1}(#2)}
\newcommand{\oprotb}[2]{\overline{\operatorname{prot}}!_{#1}(#2)}

% theorems
\theoremstyle{plain}
\newtheorem{thm}{Theorem}[chapter]
\newtheorem{lemma}[thm]{Lemma}
\newtheorem{prop}[thm]{Proposition}
\newtheorem{conj}[thm]{Conjecture}
\newtheorem{claim}[thm]{Claim}
\newtheorem{cor}{Corollary}[thm]

\theoremstyle{definition}
\newtheorem{dfn}[thm]{Definition}
\newtheorem{con}[thm]{Construction}
\newtheorem{ex}[thm]{Example}
\newtheorem{_game}[thm]{Game}
\newtheorem{_algo}[thm]{Algorithm}

\theoremstyle{remark}
\newtheorem{rmk}[thm]{Remark}
\newtheorem*{ntn}{Notation}

% operators
\def\<{\langle}
\def\>{\rangle}
\newcommand{\xcong}[1]{\overset{#1}{\cong}}
\newcommand{\xequiv}[1]{\overset{#1}{\equiv}}
\newcommand{\xequals}[1]{\overset{#1}{=}}
\def\cind{\xequiv{\text{c}}}

\def\into{\hookrightarrow}
\def\onto{\twoheadrightarrow}
\def\from{\leftarrow}
\def\To{\Rightarrow}
\def\From{\Leftarrow}
\newcommand{\xto}[1]{\xrightarrow{#1}}
\newcommand{\xinto}[1]{\xhookrightarrow{#1}}

\DeclareMathOperator{\id}{id}
\DeclareMathOperator{\Hom}{Hom}
\DeclareMathOperator{\pr}{pr}

% crypto stuff
\usepackage{mdframed}
\newenvironment{game}
{\begin{mdframed}\begin{_game}\mbox{}\begin{enumerate}[itemsep=-0.2em]}
				{\end{enumerate}\end{_game}\end{mdframed}}
\newenvironment{algo}
{\begin{mdframed}\begin{_algo}\mbox{}\begin{enumerate}[itemsep=-0.2em]}
				{\end{enumerate}\end{_algo}\end{mdframed}}

% alphabets
\def\cA{\mathcal{A}}
\def\cB{\mathcal{B}}
\def\cC{\mathcal{C}}
\def\cD{\mathcal{D}}
\def\cE{\mathcal{E}}
\def\cF{\mathcal{F}}
\def\cG{\mathcal{G}}
\def\cH{\mathcal{H}}
\def\cI{\mathcal{I}}
\def\cJ{\mathcal{J}}
\def\cK{\mathcal{K}}
\def\cL{\mathcal{L}}
\def\cM{\mathcal{M}}
\def\cN{\mathcal{N}}
\def\cO{\mathcal{O}}
\def\cP{\mathcal{P}}
\def\cQ{\mathcal{Q}}
\def\cR{\mathcal{R}}
\def\cS{\mathcal{S}}
\def\cT{\mathcal{T}}
\def\cU{\mathcal{U}}
\def\cV{\mathcal{V}}
\def\cW{\mathcal{W}}
\def\cX{\mathcal{X}}
\def\cY{\mathcal{Y}}
\def\cZ{\mathcal{Z}}

\def\AA{\mathbb{A}}
\def\BB{\mathbb{B}}
\def\CC{\mathbb{C}}
\def\DD{\mathbb{D}}
\def\EE{\mathbb{E}}
\def\FF{\mathbb{F}}
\def\GG{\mathbb{G}}
\def\HH{\mathbb{H}}
\def\II{\mathbb{I}}
\def\JJ{\mathbb{J}}
\def\KK{\mathbb{K}}
\def\LL{\mathbb{L}}
\def\MM{\mathbb{M}}
\def\NN{\mathbb{N}}
\def\OO{\mathbb{O}}
\def\PP{\mathbb{P}}
\def\QQ{\mathbb{Q}}
\def\RR{\mathbb{R}}
\def\SS{\mathbb{S}}
\def\TT{\mathbb{T}}
\def\UU{\mathbb{U}}
\def\VV{\mathbb{V}}
\def\WW{\mathbb{W}}
\def\XX{\mathbb{X}}
\def\YY{\mathbb{Y}}
\def\ZZ{\mathbb{Z}}

\def\kk{\Bbbk}
\def\ind{\mathbbm{1}}

% itemize spacing
% https://tex.stackexchange.com/a/650378
\let\tempone\itemize
\let\temptwo\enditemize
\renewenvironment{itemize}{
\tempone\setlength{\itemsep}{-1ex}}
{\temptwo}

%% PROTOCOLS %%
% \createprocedurebox{pro}{center, boxed}{}{}{}
% \newsavebox{\mybox}
% \newenvironment{protocol}[1]
% {\begin{lrbox}{\mybox}\procedure{#1}\begin{minipage}{0.9\textwidth}}
% {\end{minipage}\end{lrbox}\fbox{\usebox{\mybox}}}

\tikzset{
    triple/.style args={[#1] in [#2] in [#3]}{
        draw,#1,preaction={draw,#2,preaction={draw,#3}}
    }
}   
\tikzset{
  doublebraid/.style={
    triple={[line width=\thickness,white] in [line width=3*\thickness,black] in
    [line width=7*\thickness,white,text=black]}
  }
}

\tikzset{
  functor/.style args={#1}{
    rounded corners,fill=#1,fill opacity=.5,draw opacity=.5
  }
}
\tikzset{
  multimorphism/.style = {regular polygon, regular polygon sides=3,draw,shape border rotate=-90}
}

\tikzset{
  triangle/.style = {draw, regular polygon, regular polygon sides=3, yscale=0.7},
  rotated/.style = {shape border rotate=180},
  dashedstate/.style = {triangle, rotated, dashed},
}
% \def\multimorphism[#1,#2][#3,#4]#5#6#7
%   \path (#1,#2)
%     node [triangle,draw,shape border rotate=-90,inner sep=0pt,label=178:$\vdots$] (a) {#5} (#1,#4)
%     node [triangle,draw,shape border rotate=-90,inner sep=0pt,label=178:$\vdots$] (b) {#6} (#3,($1/2*(#4 + #5)$))
%     node [triangle,draw,shape border rotate=-90,label=135:$a_1$,label=230:$a_n$,label=178:$\vdots$] (#7) {$\theta$} (9,2)
%     node [triangle,draw,shape border rotate=-90,inner sep=-25pt,label=178:$\vdots$] (d) {$\qquad\theta\circ(\theta_1,\ldots,\theta_n)$};
% }

\def\multimorphism[#1,#2][#3][#4][#5][#6] {
    \node[multimorphism] (#3) at (#1,#2) {$#3$};
    \draw (#3.north west) to ++(-.3,0) node[left] (#3x1) {#4};
    \draw (#3.south west) to ++(-.3,0) node[left] (#3xn) {#5};
    \draw (#3.east) to ++(.3,0) node[right] (#3y) {#6};
    \node[anchor=center,left of=#3,yshift=1mm,xshift=2mm] {$\vdots$};
  }

\def\twomultimorphism[#1,#2][#3][#4][#5][#6] {
    \node[multimorphism] (#3) at (#1,#2) {$#3$};
    \draw (#3.north west) to ++(-.3,0) node[left] (#3x1) {#4};
    \draw (#3.south west) to ++(-.3,0) node[left] (#3xn) {#5};
    \draw (#3.east) to ++(.3,0) node[right] (#3y) {#6};
  }

\def\narrowtwomultimorphism[#1,#2][#3][#4][#5][#6] {
    \node[multimorphism] (#3) at (#1,#2) {$#3$};
    \draw ([yshift=-2mm]#3.north west) to ++(-.3,0) node[left] (#3x1) {#4};
    \draw ([yshift=2mm]#3.south west) to ++(-.3,0) node[left] (#3xn) {#5};
    \draw (#3.east) to ++(.3,0) node[right] (#3y) {#6};
  }
\def\wire[#1][#2][#3] {
  \draw ($(#1.east)+(.3,0)$) to[out=0,in=180] ($(#2.#3 west)+(-.3,0)$);
}

\def\comb[#1,#2][#3,#4][#5][#6] {
  \setlength\minimummorphismwidth{6mm}
  \node[morphism] (xi1) at (#1,#2) {#5};
  \node[morphism] (xi2) at (#3,#4) {#6};
  \setlength\minimummorphismwidth{0mm}
  \draw ([xshift=2.5pt]xi1.north east) to ([xshift=2.5pt]xi2.south east);
  \node[dashedmorphism] (f) at ($(xi1.north west)!.5!(xi2.south west)$)
  {\phantom{$f$}};
  \draw (xi1.north west) -- (f.south);
  \draw (xi2.south west) -- (f.north);
  \draw (xi1.south) to ++(0,-.5);
  \draw (xi2.north) to ++(0,.5);
}

\newenvironment{dedication}
  {%\clearpage           % we want a new page          %% I commented this
   \thispagestyle{empty}% no header and footer
   \vspace*{\stretch{1}}% some space at the top
   \itshape             % the text is in italics
   \raggedleft          % flush to the right margin
  }
  {\par % end the paragraph
   \vspace{\stretch{3}} % space at bottom is three times that at the top
   \clearpage           % finish off the page
  }
