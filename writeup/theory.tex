In the main body, we have assumed standard material from a course in
computability and complexity, including function asymptotics, the notion of an
algorithm, and the complexity class $P$. We briefly overview these ideas here; a
standard text is~\cite{sipser-2013}.

\section{Asymptotics}

Function asymptotics formalize the notion of a function approximating another
function. In particular, for a pair of functions $f,g:\NN\to\RR$, we often want
to compare $f$ and $g$ on large inputs and only up to a constant factor. This is
most common in runtime analysis, the idea being that the running time of
algorithms on small inputs is less important to their overall performance than
their running time on large inputs. We formalize this notion as follows:

\begin{dfn}[Function Asymptotics]
  Let $f,g:\NN\to\RR$ be a pair of functions which are both non-negative for
  sufficiently large inputs. We say that $f$ is \emph{big-Oh of
  $g$}, written $f=O(g)$, if there exists a constant $c>0$ such that for all $n$
  sufficiently large, \[
    f(n) \leq cg(n).
  \] In this case, we also say that $g$ is \emph{big-Omega of $f$}, written $g =
  \Omega(f)$.

  If $f=O(g)$ and $g=O(f)$, we say that $f$ is \emph{big-Theta of $g$}, written
  $f = \Theta(g)$. Explicitly, this means that there exist constants $c_1,c_2>0$
  such that for all $n$ sufficiently large, \[
    c_1 f(n) \leq g(n) \leq c_2 f(n).
  \]

  If $f=O(g)$ but $f\neq\Theta(g)$, we say that $f$ is \emph{little-oh of $g$},
  written $f=o(g)$, and $g$ is \emph{little-omega of $f$}, written $g =
  \omega(f)$. Explicitly, this means that for all constants $\epsilon>0$ and all
  $n$ sufficiently large, \[
    f(n) \leq \epsilon g(n).
  \]
\end{dfn}

\begin{ntn}
  By abuse of notation, we often write $f(n) = O(g(n))$ to mean that $f$ is
  $O(g)$; for example, the statement that $n^2$ is $O(n^3)$ means that the
  function $f(n) = n^2$ is $O(g)$, where $g$ is the function $n\mapsto n^3$.
\end{ntn}

\begin{ex}We have that:
  \begin{itemize}
    \item $17n^2$ is $o(n^3)$, $\omega(n)$, and $\Theta(n^2)$;
    \item $\log n$ is $o(n)$, $\omega(1)$, and $\Theta(\ln n)$;
    \item $e^n$ is $\omega(n^k)$ for any exponent $k$;
    \item $e^{-n}$ is $o(n^{-k})$ for any exponent $k$.
  \end{itemize}

  These last two examples are especially important. No matter how big the power,
  an exponential will always dominate a polynomial for sufficiently big $n$.
  Because of the importance of polynomials in theoretical computer science, we
  say a function $f$ is \emph{negligible} if $f = o(n^k)$ for all $k$. In this
  case, we write $f = \negl$ or just $f = \negl[]$.
\end{ex}

\begin{prop}
  Big-Oh is a preorder on the set of functions $\NN\to\NN$. The induced
  equivalence relation is exactly big-Theta.
\end{prop}

In the partial order of equivalence classes under $\Theta$, $O$ behaves like
$\leq$, $o$ like $<$, and $\Theta$ like $=$. As suggested by the notation $f =
O(g)$, it is common in some contexts to treat functions as identical with their
asymptotic equivalence class.

\begin{prop}\label{thm:asymptotic closure}
  Let $f_1 = O(g_1)$ and $f_2 = O(g_2)$. Let $c$ be any positive constant. Then,
  \[
    f_1 + f_2 = O(\max\{g_1,g_2\}),\quad
    cf_1 = O(g_1),\quad\text{and}\quad
    f_1f_2 = O(g_1g_2).
  \]
  In other words, \[
    O(g_1) + O(g_2) = O(\max\{g_1,g_2\}),\quad cO(g) = O(g),\quad\text{and}\quad O(g_1)O(g_2) = O(g_1g_2).
  \]
  Identical results hold for $o$ and $\Theta$.
\end{prop}

\Cref{thm:asymptotic closure} justifies the universal practice
of dropping constants and small additive terms from asymptotics, so that for
instance $n^2 + n + \ln n = \Theta(n^2)$.

\section{Algorithms and Determinism}

Our basic notion is of an \emph{algorithm} over a finite alphabet $\Sigma$,
usually $\ZZ_2$. An algorithm $\cA$ is intuitively some set of steps which take
an input word $x$ over $\Sigma$, perform some transformations, and output
another word $\cA(x)$ over $\Sigma$. An algorithm may have certain \emph{side
effects}, such as sending a message or logging a string, and its behavior may
not be deterministic. There are several ways to formalize the notion of
algorithm---most common in cryptography are Turing machines---but we will not
need to be so precise here.

Algorithms may have multiple possible ``branches'' in their instructions.
Consider the following:

\begin{_algo}\label{alg:nondeterministic}
  On input $x$, either output $0$ or $1$.
\end{_algo}

We say that algorithms of this sort are \emph{nondeterministic}; in contrast, an
algorithm is \emph{deterministic} if its instructions do not include such
choices. In particular, we say that an algorithm $\cA$ \emph{deterministically
computes} a function $f$ if it is deterministic and, for any input $x\in
\Sigma^*$, $\cA$ outputs the value $f(x)\in\Sigma^*$. In contrast, $\cA$
\emph{nondeterministically computes} $f$ if, for any input $x$, there exists a
particular choice of branches such that $\cA$ outputs $f(x)$. Thus
\Cref{alg:nondeterministic} nondeterministically computes both
the functions $x\mapsto 0$ and $x\mapsto 1$. We sometimes view nondeterministic
algorithms as computing functions into the power set of $\Sigma^*$, so that
\Cref{alg:nondeterministic} computes the function $x\mapsto\{0,
1\}$, and we similarly sometimes write $\cA(x) = \{0, 1\}$.

An important middle ground is \emph{probabilistic} algorithms. Again, there are
many possible models, but the basic idea is that a probabilistic algorithm has
access to some source of randomness---say, an arbitrarily long string of
independent and uniform coin tosses---which it can use to choose between
branches. In this case, it is not enough for there to be some branch which
computes a specific function. Instead, we say that an algorithm \emph{computes
$f$ with bounded probability} if for any input $x$, \[
  \Pr[\cA(x) = f(x)] > \frac{2}{3},
\]where the probability is taken over the randomness of $\cA$\footnote{The
  choice of $\frac{2}{3}$ is not particularly important here---generally any constant
$c>\frac{1}{2}$ works.}. In this case, we
often think of $\cA(x)$ as a probability distribution on $\Sigma^*$.

Instead of thinking of algorithms operating directly on binary strings, we
usually think of them as operating on encodings of mathematical objects. For
example:

\begin{_algo}\label{alg:double}
  On input $x$ a natural number, output the number $2x$.
\end{_algo}

We say that \Cref{alg:double} deterministically computes $x\mapsto
2x$, even though it technically operates on encodings of naturals. While there
are many possible encodings, we assume that a reasonable encoding is chosen, so
that for instance numbers are encoded in binary, rather than unary. Such details
will not be relevant for us.

One more subtlety is important. In general, we require that the description of
any algorithm $\cA$ is finite. However, we may also consider \emph{non-uniform}
algorithms, which are sequences of algorithms $\cA = (\cA_1, \cA_2, \ldots)$
such that, on an input of length $n$, $\cA$ delegates to $\cA_n$. Non-uniform
computation is generally stronger than uniform computation, as non-uniform
algorithms may encode nonfinite information, as long as they only use finitely
much of this information for each input length and hence for each
computation\footnote{
  For instance, non-uniform algorithms may solve the halting problem (which asks
  whether an input algorithm $\cM$ eventually terminates), which is uniformly
  undecidable. In particular, since there are only finitely many Turing machines
  of a given size, a non-uniform algorithm may simply encode in $\cA_n$ the
  answer to the halting problem for each Turing machine of length $n$.
}.

\section{Complexity Theory}

Each algorithm has an associated \emph{running time}, which is informally the
number of steps the algorithm takes on a given input. In particular, for an
algorithm $\cA$, we say that its running time is the function $T_\cA: \NN\to\NN$
which takes any natural number $n$ to the maximum number of steps $\cA$ takes to
terminate on any input of length $n$. Of course, this notion is not yet precise,
as we don't know what a ``step'' is, but it is easy to make precise in any
standard model of computation.

In general, the running time may depend on the formal model of computation in
which the algorithm is constructed, but the \emph{complexity-theoretic
Church-Turing thesis} states that ``reasonable'' models of classical computation
recover the same inhabitants of sufficiently robust complexity classes, in
particular of those we are about to define. This hypothesis is a heuristic, but
has been born out in practice.

\begin{dfn}[polynomial-time; P, NP]
  An algorithm $\cA$ is \emph{polynomial-time} if $T_\cA = O(n^k)$ for some
  constant $k$. The class P consists of all functions which are
  deterministically computable by polynomial-time algorithms. The class NP
  consists of all functions which are nondeterministically computable by
  polynomial-time algorithms\footnote{
    In fact, we have defined here the classes FP and FNP of polynomially- and
    nondeterministically-polynomially-computable \emph{function problems}.
    Formally, P and NP are classes of \emph{decision problems}, which are just
    subsets $L$ of $\Sigma^*$---the algorithm must output $1$ if its input is in
    $L$, and $0$ otherwise. Function and decision problems are extremely closely
    related---for instance, P = NP if and only if FP = FNP---and we will not
    distinguish between them here.
  }.
\end{dfn}

The general idea is that polynomial-time algorithms are ``efficient in
practice.'' It may sometimes occur that the constant factors or the exponent are
so large as to render the algorithm practically useless, but in most cases
functions in P are efficiently solvable for practical applications, including
cryptography. We can now state the most important open problem in computer
science:

\begin{conj}
  We have that P $\neq$ NP.
\end{conj}

While a proof seems completely out of reach, this conjecture is widely
believed, and as we will see is necessary for all of modern cryptography; we
will assume it here. An introduction to the modern state of P vs. NP
is~\cite{anderson-2017}.

Formalizing probabilistic complexity classes is slightly more subtle. Consider
the following case:
\begin{_algo}
  On input $x$, output $1$ with probability $1 - 2^{-|x|}$; otherwise count from
  $0$ to $2^{|x|}$ and then output $1$.
\end{_algo}

While this algorithm is almost always polynomial-time, it is not polynomial-time
when it takes the second branch. The point is that for probabilistic algorithms,
$T_{\cA}(n)$ is a probability distribution, not just a fixed number. For our
purposes, we require that the algorithm \emph{always} runs in polynomial time.
As such:

\begin{dfn}[probabilistic polynomial-time; BPP]
  A probabilistic algorithm is \emph{probabilistic polynomial-time} if, for any
  choice of random bits, $T_{\cA} = O(n^k)$ for some constant $k$. The class BPP
  consists of all functions which are computable with bounded probability by a
  probabilistic polynomial-time algorithm.
\end{dfn}

For non-uniform algorithms, the situation is also slightly more complicated. In
particular, it is too much to allow the machines to be arbitrarily large, as
they could simply encode lookup tables for every possible input. As such, we ask
that the size of each machine is polynomially bounded.

\begin{dfn}[non-uniform polynomial-time; P/poly]
  A non-uniform algorithm $\cA = (\cA_1, \cA_2, \dots)$ is
  \emph{polynomial-time} if $T_\cA = O(n^k)$ for some constant $k$ and the size
  of each $\cA_n$ is $O(n^k)$ for some constant $k$ independent of $n$. The
  class P/poly consists of all functions which are computable by non-uniform
  polynomial-time algorithms.
\end{dfn}

Non-uniform probabilistic algorithms are similarly defined.

\begin{thm}[Adelman's theorem]\label{thm:adelman}
  We have that BPP $\subseteq$ P/poly.
\end{thm}
