% vim:ft=tex
% TODO: some kind of intro thingy idk

Cryptography is the mathematical study of secure computation. In a computation,
we want to use \emph{protocols} to transform \emph{resources}. For the
computation to be secure, it must successfully resist \emph{attacks} by
\emph{adversaries}. We will make all of these notions precise, but first we
discuss a motivating example.\footnote{Many more examples can be found in any
	introductory text on cryptography, such as~\cite{katz-lindell-2014, rosulek-2021,
		pass-shelat-2020}.}

\section{Commitment Protocols}

Suppose we want to build an online rock-paper-scissors game. Two players, Alice
and Bob, should both be able to send each other a move and so determine the
winner. However, something needs to prevent the players from waiting until after
they learn the other's move to choose their own move. This is an ideal use-case
for a \emph{commitment protocol}.

Informally, a commitment protocol proceeds as follows. The sender $S$ has a
message $m$ that they wish to commit to---in our case, $m$ is one of
$\{R,P,S\}$. In the \emph{commit phase}, they send a commitment $c$ to the
receiver $R$. At some later time, $S$ may reveal $m$---plus maybe some auxillary
data, for example their random bits---to $R$, in which case $R$ should be able
to verify that $c$ was indeed a commitment to $m$.

We can formalize commitment schemes as follows:

\begin{dfn}[commitment protocol]\label{def:commitment protocol}
	A \emph{commitment protocol} consists of the following data:
	\begin{itemize}
		\item the \emph{message space} $\cM$ and \emph{commit space} $\cC$;
		\item a pair of families of probabalistic interactive algorithms\footnote{We
			      have been imprecise about our formal notion of algorithm. For our
			      purposes in this chapter, Turing machines suffice; we will be more
			      precise about this later. In particular, it will be important that we
			      consider the security parameter as indexing a family of algorithms, rather
			      than as a unary input to each algorithm as is common in the literature.}
		      $S_n$ and $R_n$ (indexed by the
		      \emph{security parameter} $n\in\NN$), respectively called the \emph{sender}
		      and \emph{receiver}, such that:
		      \begin{itemize}
			      \item in the \emph{commit phase}, $S_n$ gets $m\in\cM$ and returns
			            $c\in\cC$, while $R_n$ returns $\bot$ or $\top$;
			      \item in the \emph{reveal phase}, $S_n$ gets $m\in\cM$, while $R_n$
			            gets $c\in\cC$, returning $\bot$ or $m'\in\cM$.
		      \end{itemize}
	\end{itemize}
\end{dfn}

\begin{ntn}
	Let $S$ and $R$ be interactive algorithms as in \Cref{def:commitment protocol}.
  \begin{itemize}
		\item	We will write $\alg{Com}_S^R(m)$ for the output of $S$
		      in the commit phase with $S$ getting input $m\in\cM$, or $\bot$ if $R$
		      returns $\bot$.
		\item We will write $\alg{Rev}_S^R(m, c)$
		      for the output of $R$ in the reveal phase with $S$ getting input
		      $m\in\cM$ and $R$ getting input $c\in\cC$.

		      Where $S$ and $R$ are clear, we may omit the respective annotations.
	\end{itemize}
\end{ntn}

\noindent
It is worth elaborating on the ubiquitous notion of a \emph{security parameter}, as
seen in \Cref{def:commitment protocol}.\todo{TODO: this}

We would like the commitment scheme to be correct, in that when the parties
behave honestly according to the protocol, the receiver returns the correct
message. Formally:
\begin{dfn}
	A commitment protocol $(S_n,R_n)$\footnote{When $n$ is unbound, we use this
		notation to indicate a family of pairs $\{(S_n,R_n):n\in\NN\}$; when $n$ is bound it refers
		to the specific pair $(S_n,R_n)$.} is \emph{correct} if for all $n\in\NN$ and
	$m\in\cM$, \[
		\Pr[\alg{Rev}_{S_n}^{R_n}(m, \alg{Com}_{S_n}^{R_n}(m)) = m] = 1.
	\]
\end{dfn}

\noindent
Given a commitment protocol $(S_n,R_n)$, we should be able to define a family of
algorithms for our rock-paper-scissors game as follows. Let $\cM = \{R,P,S\}$
and let $W: (\bot\sqcup\cM)^2\rightarrow\{1,0,-1,\bot\}$ compute whether the
first argument beats, ties, or loses to the second, propagating any $\bot$s.

% TODO: the notation here is a little ambiguous; in particular it's not completely
% clear who has views into what. We could clean it up at the cost of notational
% overhead,but it's also pretty clear from context.
\begin{prot}{Rock-Paper-Scissors}\label{prot:rock-paper-scissors}
	\begin{enumerate}[itemsep=-0.2em]
		\item $A_n$ receives input $a\in\cM$; $B_n$ receives input $b\in\cM$.
		\item $A_n$ acts as $S_n$ and $B_n$ as $R_n$ to compute $c_a = \alg{Com}_{S_n}^{R_n}(a)$.
		\item $A_n$ acts as $R_n$ and $B_n$ as $S_n$ to compute $c_b = \alg{Com}_{S_n}^{R_n}(b)$.
		\item $A_n$ acts as $S_n$ and $B_n$ as $R_n$ to compute $a' = \alg{Rev}_{S_n}^{R_n}(a,c_a)$.
		\item $A_n$ acts as $R_n$ and $B_n$ as $S_n$ to compute $b' = \alg{Rev}_{S_n}^{R_n}(b,c_b)$.
		\item $A_n$ returns $W(a,b')$.
		\item $B_n$ returns $W(a',b)$.
	\end{enumerate}
\end{prot}

\begin{ntn}
	Given some fixed commitment protocol, we will write
	$\alg{RPS}_{A}^{B}(a,b)$ for the results returned by $A$ and $B$,
	respectively. If $A$ or $B$ are honest, we will omit the corresponding
	annotation.
\end{ntn}

\noindent
As with commitment, we can define correctness of this protocol.
\begin{dfn}
	\Cref{prot:rock-paper-scissors} is \emph{correct} relative to a commitment
	protocol $(S_n,R_n)$ if for all $a,b\in\{R,P,S\}$,
	\[
		\Pr[\alg{RPS}(a, b) = (W(a,b), W(a,b))] = 1.
	\]
\end{dfn}

\begin{prop}
	\Cref{prot:rock-paper-scissors} is correct relative to any correct commitment
	protocol.
\end{prop}

\begin{proof}
	Fix a correct commitment protocol. Then \begin{align*}
		 & \Pr[\alg{RPS}(a,b) = (W(a,b),W(a,b))]                                     \\
		 & = \Pr[W(a,b') = W(a,b)\text{ and }W(a',b) = W(a,b)]                       \\
		 & = \Pr[W(a,b') = W(a,b)]\cdot\Pr[W(a',b) = W(a,b)]                         \\
		 & \geq \Pr[b'=b]\cdot\Pr[a'=a]                                              \\
		 & = \Pr[\alg{Rev}(b,c_b)=b]\cdot\Pr[\alg{Rev}(a, c_a) = a]                  \\
		 & = \Pr[\alg{Rev}(b,\alg{Com}(b))=b]\cdot\Pr[\alg{Rev}(a,\alg{Com}(a)) = a] \\
		 & = 1.\qedhere
	\end{align*}
\end{proof}

\noindent
Furthermore, at least in this case, it was easy to define notions of correctness
that compose. Our task now is to define security of commitment and
rock-paper-scissors such that, whenever a commitment scheme is secure,
\Cref{prot:rock-paper-scissors} is likewise secure.

% \section{Desiderata}

% \todo{I think we should say something like this at some point, but to be honest
% 	I think I'm way too inexperienced to have confidence in what I'm saying. These
% 	all seem like reasonable considerations, but I'm not sure I want to get into
% 	this kind of argument in what should be a largely technical paper.}

% In order to evaluate security definitions, we need to know what the properties
% of a good definition are. There are many possible factors; we outline some that
% will be relevant here.

% First, as with all definitions in computer science, it must capture the
% intuitive notions we wish to formalize. Formal security proofs are no good if
% our notion of security is meaningless. There is, of course, no formal way to
% prove that a definition is correct in this sense, so on some level we must try
% enough examples to build trust that we have not missed any significant
% incongruities between our mathematical model and the real world.

% Second, the definitions should be amenable to proof. Complex definitions with
% many moving parts, even when they have better technical properties, are not
% conducive to good mathematical practice. They make proofs hard to follow and
% trust, 

\section{Composition}

We are centrally concerned in this thesis with the properties of security
definitions under composition. By composition, we informally mean
stitching-together various protocols to form some larger protocol\footnote{The
	exact formal definition depends on the underlying model of computation, which
	we have deliberately left unspecified.}. We will say that a security
definition \emph{composes} if, whenever each of the smaller protocols is
secure, the larger protocol is also secure.

There are two separate types of composition to consider. In \emph{sequential}
(or \emph{vertical}) composition, only one of the small protocols is running at
any given time. In the more general setting of \emph{parallel} (or
\emph{horizontal}) composition, the parties involved may be engaged at various
stages of many small protocols at once. There are several in-between notions to
consider. For example, one can consider bounded-width composition, where the
number of algorithms running in parallel is bounded by some function of the
security parameter.

\Cref{prot:rock-paper-scissors} is also somewhere in between sequential and
parallel composition. There are two commitment protocols running simultaneously,
but each phase of each protocol is running sequentially: while the parties are
actively engaged in a specific commit or reveal phase, they are doing nothing
else. As we will see when we discuss simulation security, working with
multi-phase protocols like commitment adds a layer of complexity to many
frameworks.

\section{Game-Based Security}

In \emph{game-based} approaches to security, we define security by determining
the winner of an abstract game. Here, we encode specific properties we want the
algorithm to have, and say that an adversary wins the game if they can break the
property. In standard approaches to
commitment\footnote{See~\cites[Section 4.4.1]{goldreich-2001}.}, there are two
desirable properties. \emph{Hiding} means that the receiver should not learn
anything about the message until the reveal phase. \emph{Binding} means that the
sender should not be able to trick the receiver into anything other than the
committed message. Formally:

\begin{dfn}\label{def:game commitment}
	A commitment scheme $(S_n, R_n)$ is \emph{game-secure} if the following hold.
	\begin{itemize}
		\item \emph{Hiding:} consider the following game against a family of
		      adversaies $R'_n$.
		      \begin{game}\label{game:hiding}
			      \item $R'_n$ outputs $m_0,m_1\in \cM$.
			      \item A random bit $b\in\{0,1\}$ is chosen; $m_b$ is given to $S_n$.
			      \item $S_n$ and $R'_n$ participate in the commit phase.
			      \item $R'_n$ outputs a guess $b'\in\{0, 1\}$. $R'_n$ wins if $b' =
				      b$.
		      \end{game}
		      A commitment scheme is \emph{hiding} if for any family of
		      probabalistic polynomial-time algorithms $R'_n$, \[
			      \Pr[R'_n\text{ wins \Cref{game:hiding}}] \leq \frac{1}{2} + \negl.
		      \]
		      The idea is that $R'_n$ participates in the commit phase for a randomly
		      chosen message $m_b$, and then tries to guess $b$; they should be able
		      to guess no more than negligibly\footnote{A function is \emph{negligible},
			      written $\negl$, if it is asymptotically smaller than any rational
			      function in $n$.} better than random.
		\item \emph{Binding:} consider the following game against a family of
		      adversaires $S'_n$.
		      \begin{game}\label{game:binding}
			      \item $S'_n$ outputs $m\in\cM$.
			      \item $S'_n$ and $R_n$ participate in the commit phase.
			      \item A random bit $b\in\{0,1\}$ is chosen and given to $S'_n$.
			      \item $S'_n$ and $R_n$ participate in the reveal phase.
			      \item If $b = 0$, $S'_n$ wins if $R_n$ outputs $m$. If $b = 1$,
			      $S'_n$ wins if $R_n$ outputs any $m'\notin \{m, \bot\}$.
		      \end{game}

		      A commitment scheme is \emph{binding} if for any family of
		      probabalistic polynomial-time algorithms $S'_n$, \[
			      \Pr[S'_n\text{ wins \Cref{game:binding}}] \leq \frac{1}{2} + \negl.
		      \]
		      The idea is that $S'_n$ first makes a commitment, and then learns
		      whether they want to lie to $R_n$; they should be able to do no more
		      than negligibly better than guessing which of the two messages to
		      commit to before learning the random bit.
	\end{itemize}
\end{dfn}

\noindent
Similarly, we can define security of \Cref{prot:rock-paper-scissors} by having
the adversary play against an honest party who moves according to some
probability distribution.

\begin{dfn}
	\Cref{prot:rock-paper-scissors} is \emph{game-secure} relative to a commitment
	protocol $(S_n,R_n)$ if the following hold for any probability distribution
	$P$ on $\cM$:
	\begin{itemize}
		\item \emph{A-security}: For any family of PPT\footnote{Probabalistic
			      polynomial-time, though this definition can be made relative to any class
			      of adversarial algorithms.} algorithms $A'_n$,\[
			      \Pr[\pr_2(\alg{RPS}_{A'_n}(a, b)) = 1] \leq \max_{m\in\cM}P(m) + \negl,
		      \]
		      where the randomness is over uniform choice of $a$ and choice of $b$
		      according to $P$.
		\item \emph{B-security}: For any family of PPT algorithms $B'_n$,\[
			      \Pr[\pr_1(\alg{RPS}^{B'_n}(a, b)) = -1] \leq \max_{m\in\cM}P(m) + \negl,
		      \]
		      where the randomness is over choice of $a\in\cM$ according to $P$ and
		      uniform choice of $b$.
	\end{itemize}
\end{dfn}

\noindent
Observe that when the adversary controls $A$, we care only that $B$ outputs a
fair view of the game, and when the adversary controls $B$, we only care that
$A$ outputs a fair view of the game: we can never prevent the adversary from
just outputting that they win.

\vspace{1em}

Surprisingly, as we now show, it does not hold that \Cref{prot:rock-paper-scissors} is
game-secure relative to any game-secure commitment scheme.

Let $f:\bin^*\rightarrow\bin^*$ be a \emph{one-way permutation}\footnote{That such functions exist is a
	standard cryptographic assumption; see any introductory textbook, for
	example~\cite[Section 7.1]{katz-lindell-2014}.}, i.e. an injective function
which is easy to compute but hard to invert. Formally,
this means that\begin{itemize}
	\item $f$ is poly-time computable;
	\item for any family of PPT algorithms $f'_n$, \[
		      \Pr_{x\leftarrow\bin^n}[f(f'_n(f(x))) = f(x)] \leq \negl.
	      \]
\end{itemize}

Let $h: \bin^*\rightarrow\ZZ_3$ be a \emph{ternary\footnote{Typically, by
		hard-core predicate we mean a function which outputs a Boolean; we extend the
		notion for the case of rock-paper-scissors.} hard-core predicate}\footnote{If
	one-way permutations exist, then so too do one-way permutations with hard-core
	predicates~\cite{yao-1982}.} of $f$, i.e., a
function which is easy to compute, but hard to compute ``on $f$,'' in that
\begin{itemize}
	\item $h$ is poly-time computable;
	\item for any family of PPT algorithms $h'_n$, \[
		      \Pr_{x\leftarrow\bin^n}[h'_n(f(x)) = h(x)] \leq \frac{1}{3} + \negl.
	      \]
\end{itemize}

\begin{prot}{$(f,h)$-Commitment}\label{prot:commitment}
	\newline
	\emph{Commit:}
	\begin{enumerate}[itemsep=-0.2em,topsep=-0.2em]
		\item $S_n$ gets $m\in\{R,P,S\}$ and interprets as an element of $\ZZ_3$.
		\item $S_n$ picks $x\in\bin^n$ uniformly at random.
		\item $S_n$ sends $(f(x), h(x) + m)$ to $R_n$.
		\item $S_n$ returns $(f(x), h(x) + m)$.
		\item $R_n$ returns $\top$.
	\end{enumerate}
	\vspace{1em}
	\emph{Reveal:}
	\begin{enumerate}[itemsep=-0.2em,topsep=-0.2em]
		\item $S_n$ sends $x$ to $R_n$.
		\item $R_n$ verifies the values of $f(x)$ and $h(x) + m$.
		\item If the verification succeeds, $R_n$ returns $m$; else they return $\bot$.
	\end{enumerate}
\end{prot}

\noindent
The idea is that secrecy is guaranteed by hardness of $h$, while
bindingness is guaranteed by injectivity of $f$. Indeed:

\begin{prop}
	\Cref{prot:commitment} is game-secure.
\end{prop}
\begin{proof}We need to show two things.
	\begin{itemize}
		\item \emph{Hiding:} let $R'_n$ be a family of PPT algorithms. Make a
		      family $h'_n$ which proceeds as follows:
		      \begin{enumerate}[itemsep=-0.2em]
			      \item Receive input $y\in\bin^*$.
			      \item Run $R'_n$ to get $m_0,m_1\in\{R,P,S\}$.
			      \item Send $(y, 0)$ to $R'_n$, get back a guess $b'$.
			      \item Output $-m_{b'}$.
		      \end{enumerate}

		      If $h(x) + m_b = 0$ for some $b$, then $h'_n$ guesses $h(x)$ whenever
		      $R'_n$ guesses $b$. However, since $h(x)$ must look uniformly
		      distributed, this occurs negligibly close to $\frac{2}{3}$ of the
		      time. Hence since $h'_n$ guesses right no more than negligibly more
		      than $\frac{1}{3}$ of the time, $R'_n$ guesses right no more than
		      negligibly more than $\frac{1}{2}$ of the time.
		\item \emph{Binding:} By injectivity of $f$, there is a unique $x$ such that
		      $f(x)$ is committed. Once $R_n$ gets this unique $x$, they can
		      deterministically verify $f(x)$ and $h(x)+m$, returning $\bot$ if this
		      fails. \qedhere
	\end{itemize}
\end{proof}

\noindent
This is an example of \emph{proof by reduction}, a common technique in computer
science. The idea is to reduce one problem into another by starting with a
solution to the second and showing how to solve the first. If the first problem
is known to be hard, in some precise sense, then by studying the structure of
the reduction we can learn about the hardness of the second problem. In this
case, we reduce the problem of computing the hard-core predicate into the
problem of breaking the commitment scheme. To do this, we show that if we have
some strategy $R'_n$ to attack the commitment scheme, then we can use it to
build a strategy $h'_n$ to compute the hard-core predicate. We can then show
hidingness of the commitment scheme using our assumption about the hardness of
the hard-core predicate.

Regardless, we now observe that \Cref{prot:rock-paper-scissors} is not
game-secure relative to \Cref{prot:commitment}. The problem is that
\Cref{prot:commitment} is \emph{malleable}, in that the receiver can compute the
commitment of a message related to the message committed by the receiver. In
particular, to win at rock-paper-scissors, Bob only needs to take Alice's
commitment $(y, z) = (f(x), h(x) + m)$ and return $(y, z+1) = (f(x), h(x) + m +
	1)$, which is a valid commitment to $m+1$. This does not violate secrecy, as the
receiver still cannot learn anything about the message from this new computed
commitment.

Realizing this, we could now add non-malleability to the list of security
properties in \Cref{def:game commitment}. However, this example reveals a
broader issue with game-based security definitions: they rely on ad-hoc
specifications of security properties, and we need to trust that we have thought
hard enough to ensure that the properties we have specified are enough. For this
reason, there are no general composition theorems for game-based security, and
so large composite protocols of the kind used in real-world applications have to
be proven secure by hand. In the next section, we outline a more principled
approach.


\section{Simulation-Based Security}

In \emph{simulation-based} approaches to security, we define security by
comparing a protocol in the real world to an ideal world in which the desired
resource is produced by a trusted black-box. We say a protocol is secure if a
real-world adversary has no more influence on the outcome than they would in the
ideal case. The proofs typically proceed by showing that any adversary in the
real world can be ``simulated'' by an adversary in the ideal world.

It's easy to define the ideal world for commitment protocols, using a trusted
party $T$:

\begin{prot}{Ideal Commitment}\label{prot:ideal commitment}
	\newline
	\emph{Commit:}
	\begin{enumerate}[itemsep=-0.2em,topsep=-0.2em]
		\item $\ideal{S}_n$ gets $m\in\cM$.
		\item $\ideal{S}_n$ sends $m$ to $T$.
	\end{enumerate}
	\vspace{1em}
	\emph{Reveal:}
	\begin{enumerate}[itemsep=-0.2em,topsep=-0.2em]
		\item $T$ sends $m$ to $\ideal{R}_n$.
		\item $\ideal{R}_n$ outputs $m$.
	\end{enumerate}
\end{prot}

\begin{ntn}
	As in \Cref{prot:ideal commitment}, we will generally use overlines to denote
	parties engaged in an ideal protocol.
\end{ntn}

\noindent
It remains to define the correct sense in which to compare attacks on the actual
protocol to the ideal protocol.

\begin{dfn}[computataional indistinguishability]\label{def:computataional indistinguishability}
	Two families of random functions $\{X_n: A\rightarrow B\}$ and $\{Y_n:
		A\rightarrow B\}$ are \emph{computationally indistinguishable} if, for any
	PPT algorithm $D$ and $a\in A$, we have\[
		|\Pr[D(X_n(a)) = 1] - \Pr[D(Y_n(a)) = 1]| \leq \negl.
	\]In this case, we write $X\compind Y$.
\end{dfn}

\noindent
The idea is that $D$ attempts to distinguish between $X$ and $Y$, and must fail
to do so all but negligibly often. If the combined output of some real protocol is
computationally indistinguishable from the output of the ideal protocol, then
this indicates that \emph{no} adversary can distinguish between execution of the real
protocol and the ideal protocol. As such, when we eventually prove composition
theorems, we will be able to substitute the ideal protocol in place of the real
protocol in order to prove security of some larger protocol.

\begin{ntn}
	We now let $\alg{Com}_S^R(m)$ denote the \emph{pair} of outputs of $S$ and $R$
	in the commit phase, with $S$ getting input $m$, and so too for
	$\alg{Rev}_S^R(m, c)$. Recall that $\alg{Com}_{\ideal{S}}^{\ideal{R}}$ and
	$\alg{IRev}_{\ideal{S}}^{\ideal{R}}$ refer to the ideal protocol. Note that in
	this case, we omit the trusted party $T$, who always acts according to the
	protocol.
\end{ntn}

\noindent
Combining the pair of outputs allows us to deal with issues like malleability,
where (by hidingness) the receiver's output on its own is no different from that
in the ideal protocol, but the pair of outputs may be correlated in the real
protocol but not the ideal protocol.

% TODO: actually needs to be against any adversary, ofc
\begin{dfn}
	A commitment protocol $(S_n, R_n)$ is \emph{simulation-secure} if the
	following hold:

	\begin{itemize}
		\item \emph{Secure commit}: for any family of PPT algorithms $R'_n$, there
		      exists a family of PPT algorithms $\ideal{R}'_n$ such that for any
		      $m\in\cM$,
		      \[
			      \alg{Com}_{S_n}^{R'_n}(m) \compind\alg{Com}_{\ideal{S}_n}^{\ideal{R}'_n}(m).
		      \]
		\item \emph{Secure reveal}: for any family of PPT algorithms $S'_n$,
		      there exists a family of PPT algorithms $\ideal{S}'_n$ such that for any
		      $m\in\cM$ and $c\in\cC$, \[
			      \alg{Rev}_{S'_n}^{R_n}(m, c) \compind\alg{Rev}_{\ideal{S}'_n}^{\ideal{R}_n}(m, c).
		      \]
	\end{itemize}
\end{dfn}

\noindent
To demonstrate the definition, we observe that \Cref{prot:commitment} is \emph{not}
simulation-secure.\todo{TODO: the definition is not quite right, so we can't
	finish this thought yet. We need to talk about the difficulty of proof in the
	paradigm, and also just how simulation proofs work.}
% TODO: is any protocol secure by this defintion? what if R' just outputs the
% commit and D checks whether S and R' outputted the same?

Indeed, simulation security is sufficiently strong as to obtain a general
sequential composition theorem:

\begin{thm}[\cite{micali-rogaway-1992}]\label{thm:sequential composition for simulation security}
	Let $\{f_1,\cdots,f_{p(n)}\}$ be a set of polynomially-many \emph{two-party
		functionalities}, meaning functions
	$\bin^*\times\bin^*\rightarrow\bin^*\times\bin^*$. Let $\{\pi_i\}$ be
	simulation-secure protocols for computing each of the $\{f_i\}$. Then the
	sequential composition of the $\pi_i$s is a simulation-secure protocol for
	computing the composite functionality $f_{p(n)}\circ\cdots\circ f_1$.
\end{thm}

\noindent
We interpret the functionalities $f_i$ as computational problems where two
parties, given inputs $x,y\in\bin^*$, work together to compute outputs
$\pr_1(f_i(x, y)), \pr_2(f_i(x, y))$. Such functionalities encompass a huge
class of cryptographic protocols. The theorem says that if we know how to
compute each $f_i$ in a simulation-secure way---where the ideal world involves
both parties giving their inputs to the trusted party and recieving their
outputs back---then we can compute the composite simulation-securely, as well.
For a more careful treatment of this theorem, and simulation security as a whole,
see~\cite{lindell-2017}.

% \begin{rmk}\label{rmk:polynomial composition only}
It's important to observe that
\Cref{thm:sequential composition for simulation security} works only for polynomially many functionalities. Recall that
computational indistinguishability requires only a neglible difference between
the real and ideal worlds, rather than requiring no difference. As a
consequence, the theorem cannot hold for arbitrarily many functionalities, as
the super-polynomial composition of negligible functions may fail to be
negligible. In fact, computational indistinguishability is not even a real
equivalence relation, as transitivity fails with more than polynomially many
compositions. We make an attempt to illustrate the issue in
\Cref{fig:polynomial
distinguishability}. Regardless, this is a fundamental issue: as long as our
definitions of security are computational, we should not expect a
super-polynomial composition of secure processes to remain secure.

\begin{fig}[t]
  \[
  \begin{pic}
			\node[morphism] (fk) at (0,2.5) {$f_{K(n)}$};
			\node[dashedmorphism] (dots) at (0,1.5) {};
			\node[morphism] (f1) at (0,0) {$f_1$};
			\node[morphism] (f2) at (0,.75) {$f_2$};
			\draw (f1.south) to ++(0,-.3) node[right] {};
			\draw (f2.south) to (f1.north);
			\draw (f2.north) to (dots.south) {};
			\draw (fk.south) to (dots.north) + (0,-0.2);
			\draw (fk.north) to ++(0,.3) node[right] {};

			% \node[font=\normalfont] at (1.5,1.25) {$\overset{C}{\equiv}$};

			\node (o1) at (1.5,0) {$o(n^{-d})$};
			\node (o2) at (1.5,.75) {$o(n^{-d})$};
			\node (ok) at (1.5,2.5) {$o(n^{-d})$};
			\draw[-|] (o1) to (0.75,0);
			\draw[-|] (o1) to (2.25,0);
			\draw[-|] (o2) to (0.75,.75);
			\draw[-|] (o2) to (2.25,.75);
			\draw[-|] (ok) to (0.75,2.5);
			\draw[-|] (ok) to (2.25,2.5);

			\node[morphism] (gk) at (3,2.5) {$g_{K(n)}$};
			\node[dashedmorphism] (gdots) at (3,1.5) {};
			\node[morphism] (g1) at (3,0) {$g_1$};
			\node[morphism] (g2) at (3,.75) {$g_2$};
			\draw (g1.south) to ++(0,-.3) node[right] {};
			\draw (g2.south) to (g1.north);
			\draw (g2.north) to (gdots.south) {};
			\draw (gk.south) to (gdots.north) + (0,-0.2);
			\draw (gk.north) to ++(0,.3) node[right] {};
		\end{pic}
  \]
  \caption{Even if $K(n)$ processes are computationally
  indistinguishable---meaning the adversary has only $o(n^{-d})$ chance to
  distinguish them---if $K(n)$ is super-polynomial, the composition of these
  processes may be distinguishable.\label{fig:polynomial distinguishability}}
\end{fig}
% \end{rmk}

Regardless, while simulation security is generally a stronger guarantee than
game security, it is not perfect. In 1996, Goldreich and Krawczyk constructed a
protocol for \emph{zero-knowledge proof} which is simulation-secure, but is not
secure under parallel composition~\cite{goldreich-krawczyk-1996}. The
construction is quite artificial, and we are not aware of any natural examples
in which simulation security fails to compose. Nonetheless, for the purpose of
provable security, we need a more powerful framework. There are several options,
but the most popular by far is \emph{universal composability}.

\section{Universally Composable Cryptography}

Universally composable cryptography, due to Ran Canetti, is a framework for
writing composible proofs of security~\cite{canetti-2000}. While a full
treatment is far outside our scope, we give some intuition here, and conclude
with a discussion of some drawbacks of the framework.

UC maintains the same basic idea as simulation security---the actual execution
of the protocol is compared to some ideal world in which a trusted, unbounded
third party is available. The main difference is how the ideal world is compared
to the actual protocol. In simulation security, we use the straightforward
notion of computational indistinguishability of outputs---at the end of the
protocol, both parties output some value, and we compare the distribution of
outputs in the ideal world to the real world. Even with commitment, we saw that
this poses some difficulties when we want the protocol to satisfy some property
midway through its execution.

In contrast, in universal composability, the adversary freely interacts with a
separate (poly-time) algorithm, the \emph{environment}. To be UC-secure, it
should be impossible for the environment to distinguish between the ideal
protocol and the real protocol \emph{at any point during the execution}. This
much stronger guarantee allows us to obtain a general parallel composition
theorem, the intuition being that the environment could include artbirary other
protocols, running simultaneously with the protocol under proof, so UC security
implies security under even (polynomial-width) parallel composition.

On the other hand, this framework comes with onerous proof buderns. Whereas in
the simulation paradigm, the ideal-world adversary must simulate interaction with
the actual protocol, in the UC paradigm, the ideal-world adversary must also
simulate the real-world adversary's interaction with the environment. Because
this interaction is not constrained other than by computational bounds on the
environment, even the simplest UC proofs require complex arguments about the
flow of information through the protocol.

As a consequence, many cryptographers find UC proofs hard to trust. The
framework is fundamentally tied to certain technical choices, such as the
computational model of turing machines, meaning it requires modification to
deal with, for example, quantum
cryptography~\cite{hofheinz-mullerquade-2003,unruh-2010}. The proofs tend to be
extremely technical, often not even included in published work but instead left
to supplemental material. The framework has been modified seven times since its
introduction in 2000, including as recently as 2020~\cite{canetti-2020}, and many of
these versions have introduced incompatible technical changes. The cumulative
result of all of this is that, while UC is undoubtedly the most powerful
framework for composable proof in common use today, it is not as widespread as
one might expect.

\section{Conclusion}

In this chapter, we considered three frameworks for security definitions in
cryptography: game-based security, simulation-based security, and universal
composability. Each of these frameworks exhibits tradoffs between complexity,
composability, and ease of proof. While game-based proofs are simple to write,
game-based definitions tend to be ad-hoc, making the definitions hard to trust
and composition theorems impossible to obtain. On the other hand, universal
composability obtains extremely powerful composition theorems, and leaves no
room for error in writing definitions, but leads to complex proofs ridden with
technical details. Simulation-based security lies somewhere in the middle, with
good properties under sequential composition, but falls short of the generality
of the UC framework.

This situation calls out for a new framework which takes advantage of powerful
mathematical abstractions to simplify proofs, while maintaining the composition
guarantees of universal composability. We believe that category theory is the
right mathematical abstraction on which to base such a framework. In the next
chapter, we will introduce the basic notions of this field, motivated by their
potential relationship to cryptography.
