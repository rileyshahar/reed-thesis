% vim:ft=tex
% TODO: some kind of intro thingy idk

\section{Resources, Protocols, Adversaries, Attacks}

Cryptography is the mathematical study of secure computation. In a computation,
we want to use \emph{protocols} to transform \emph{resources}. For the
computation to be secure, it must successfully resist \emph{attacks} by
\emph{adversaries}. We will make all of these notions precise, but first we
discuss some motivating examples. Many more examples can be found in any
introductory text on cryptography, such as~\cite{katz-lindell-2014, rosulek-2021,
	pass-shelat-2020}.

\begin{ntn}
	We first need some notation for these examples.
	\begin{itemize}
		\item We write $x\sample X$ to mean $x$ is drawn from the distribution
		      $X$; when $X$ is a set we affix the uniform distribution.
		\item We write $\oplus$ for the bitwise XOR (addition modulo $2$) of two
		      binary strings.
	\end{itemize}
\end{ntn}

\begin{ex}[The One-Time Pad]\label{ex:one-time pad}
	Two parties, Alice and Bob, have the same \emph{private key}
	$k\sample\bin^\ell$ and can communicate over an \emph{insecure channel} $C$.
	Alice has a \emph{message} $m\in\bin^\ell$ that she wants to send to Bob.
	These define the resources required for the following protocol:

	\procedureblock[linenumbering]{One-Time Pad}{
		\text{Alice insecurely sends $c := m \oplus k$ to Bob}. \\
		\text{Bob decodes the message as $c\oplus k$}.
	}

	\noindent
	A proof of \emph{correctness} of this protocol should say, roughly, that Bob
	recieves the same message $m$ that Alice sent.
	% TODO: mention where we formalize these proofs

	Since the channel is insecure, a third party, Eve, may observe the ciphertext
	$c$ sent by Alice. Eve is an \emph{adversary} against the protocol. A proof of
	\emph{security} of the protocol should say, roughly, that Eve cannot learn
	anything about $m$ from observing $c$.
\end{ex}

\noindent
In other words, one-time pad is a protocol with two open input ports and one
open output port\footnotemark. When ``hooked up'' to an insecure channel and a
shared string of length $\ell$, it produces a secure channel of capacity $\ell$.
We can notate this as in \Cref{fig:one-time pad}.

% This perspective is different from the one of texts
% like~\cite{katz-lindell-2014}, which emphasize the API of the protocol as
% a trio of algorithms for key generation, encoding, and decoding.

\footnotetext{There are of course many details which we do not communicate in
	this diagram, for instance that the key must be uniform random.}

\begin{ex}[Pseudorandom Generators]\label{ex:pseudorandom generators}
	A \emph{pseudorandom generator} should take some small input string, its
	\emph{seed}, and produce a longer output string, in such a way the its outputs
	cannot be distinguished from uniform random strings. Again, we will make this
	formal soon, but for now we express this in \Cref{fig:pseudorandom generator}.
\end{ex}

\begin{figure}
	\centering
	\begin{subfigure}{0.5\textwidth}
		\centering
		\begin{tikzpicture}
			\node (C) at (-2, -2) {\text{insecure channel}};
			\node (k) at (2, -2) {\text{shared string of length $\ell$}};
			\node[box=1/0/2/0, minimum width=2cm] (OTP) at (0, 0) {\text{One-Time Pad}};
			\node (Out) at (0, 2) {\text{secure channel of capacity $\ell$}};
			\wires[looseness=1]{
				Out = { south = OTP.north },
				C = { north = OTP.south.1 },
				k = { north = OTP.south.2 },
			}{}
		\end{tikzpicture}
		\caption{The One-Time Pad.}
		\label{fig:one-time pad}
	\end{subfigure}
	\begin{subfigure}{0.4\textwidth}
		\begin{tikzpicture}
			\node (r) at (0, 2) {\text{string of length $\ell$}};
			\node (s) at (0, -2) {\text{string of length $r < \ell$}};
			\node[box=1/0/1/0, minimum width=2cm] (G) at (0, 0) {\text{Pseudorandom Generator}};
			\wires[looseness=1]{
				r = { south = G.north },
				s = { north = G.south },
			}{}
		\end{tikzpicture}
		\caption{A pseudorandom generator.}
		\label{fig:pseudorandom generator}
	\end{subfigure}
	\caption{Basic cryptographic protocols.}
\end{figure}

\noindent
A limitation of one-time pad is that it needs as long a key as the channel it
produces. We should now be able to fix this by \emph{composing} a pseudorandom
generator with the one-time pad, as in \Cref{fig:composed}.

\begin{figure}[h]
	\centering
	\begin{tikzpicture}
		\node (C) at (-2, -2) {\text{insecure channel}};
		\node (k) at (2, -2) {\text{shared string of length $\ell$}};
		\node (s) at (2, -6) {\text{shared string of length $r < \ell$}};
		\node[box=1/0/2/0, minimum width=2cm] (OTP) at (0, 0) {\text{One-Time Pad}};
		\node[box=1/0/1/0, minimum width=2cm] (G) at (2, -4) {\text{Pseudorandom Generator}};
		\node (Out) at (0, 2) {\text{secure channel of capacity $\ell$}};
		\wires[looseness=1]{
			Out = { south = OTP.north },
			C = { north = OTP.south.1 },
			k = { south = G.north,
					north = OTP.south.2
				},
			s = { north = G.south },
		}{}
	\end{tikzpicture}
	\caption{Composing One-Time Pad with a pseudorandom generator.}
	\label{fig:composed}
\end{figure}

This scheme raises an immediate issue.  we need to know that our notions of
correctness and security \emph{compose}: if one-time pad and the pseudorandom
generator are both correct or both secure, is the composite procotol correct or
secure? Notice that, in the composite diagram, Alice and Bob must both
separately use the pseudorandom generator to convert their seed into a longer
key. We thus need composition both \emph{in sequence}---``vertically'' in the
diagram---and \emph{in parallel}---``horizontally'' in the diagram.

\section{Formulating Cryptographic Protocols}

Cryptographic protocols are procedures for transforming resources. To do math
with them, we need a precise definition of what resources they require and
produce. For instance, one-time pad is a \emph{private-key encryption scheme}.

% TODO: there are a lot of subleties with this definition, e.g. if you want it
% to be in PPT then you need Enc/Dec to take k as a parameter (or 1^n in unary)
\begin{dfn}
	A \emph{private-key encryption scheme} over a \emph{message space} $\cM$ and
	\emph{key space} $\cK$ consists of the following data:
	\begin{itemize}
		\item a probabalistic \emph{key generation algorithm}\footnotemark{}
		      $\alg{Gen}: 1^*\rightarrow \cK$;
		      \footnotetext{By ``algorithm'',
			      we could have several different notions of computation in mind. Usually we
			      don't require this level of specificity, but it suffices to consider Turing
			      machines.}
		\item a family of \emph{encryption algorithms} $\alg{Enc}_k: \cM\rightarrow
			      \cM$ indexed by $\cK$;
		\item a family of \emph{decryption algorithms} $\alg{Dec}_k:
			      \cM\rightarrow\cM$ indexed by $\cK$.	\end{itemize}
\end{dfn}

This definition allows us to straightforwardly formulate the correctness
of such a scheme:

\begin{dfn}
	A private-key encryption scheme is \emph{information-theoretically correct} if
	for all $k\in \cK$ and $m\in \cM$, \[
		\alg{Dec}_k(\alg{Enc}_k(m)) = m.
	\]
\end{dfn}

\noindent
In practice, however, we often want to allow for a negligible probability of
failure. In this case, we define:

\begin{dfn}
	A private-key encryption scheme is \emph{computationally correct} if
	for all $k\in \cK$ and $m\in \cM$, \[
		\Pr[\alg{Dec}_k(\alg{Enc}_k(m)) \neq m] = \negl[|k|],
	\]where a function is \emph{negligible}, written $\negl[]$, if it is
	asymptotically smaller than any rational function.
\end{dfn}
\section{Game-Based Security}

In \emph{game-based} approaches to security, we define security by determining
the winner of an abstract game.

\section{Simulation-Based Security}

In \emph{simulation-based} approaches to security, we define security by
comparing a protocol in the real world to an ideal world in which the desired
resource is produced by a trusted black-box.
