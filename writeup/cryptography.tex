[TODO: An introduction to the chapter;
cite~\cite{katz-lindell-2014, pass-shelat-2020, rosulek-2021}.]

\section{Foundations}

\subsection{One-way functions}

Many cryptographic protocols rely on \emph{one-way functions}, which are
informally functions that are easy to compute, but hard to invert. The former
notion is easy to formalize in terms of time complexity, but the latter is more
difficult. We typically ask that any ``reasonably efficient'' algorithm---called
the \emph{adversary}---attempting to invert the function has a negligible chance
of success. (Recall that a function $f$ is \emph{negligible} if $f = o(n^{-k})$
for every $k$, in which case we write $f = \negl$ or just $f = \negl[]$.)

% The difficulty is in determining which adversaries are ``reasonable''. We
% generally ask that adversaries are non-uniform probabilistic polynomial-time
% algorithms. The non-uniformity primarily serves to simplify proofs, by allowing
% us to not worry about the size of the adversary.

\begin{ntn}
  We will use PPT as shorthand for probabilistic polynomial-time, and the term
  \emph{adversary} for non-uniform PPT algorithms.
\end{ntn}

\begin{dfn}[one-way function]\label{def:one-way function}
  A function $f$ is \emph{one-way} if:
  \begin{itemize}
    \item (easy to compute) $f$ is PPT-computable;
    \item (hard to invert) for any adversary $\cA$, natural number $n$, and
      uniform random choice of input $x$ such that $|x| = n$, \[
        \Pr[f(\cA(1^n, f(x))) = f(x)] = \negl.
      \]
  \end{itemize}
  Note that $|x|$ here is \emph{not} the absolute value, but is instead the
  length of $x$ as a binary string: if $x$ is a number, then by encoding in
  binary have that $|x| = \Theta(\log_2 x)$.
\end{dfn}

The idea is that, given $y = f(x)$, $\cA$ attempts to find some $x'$ such that
$f(x') = y$. If some adversary can do this with non-negligible probability, then the
function is not one-way. While the probability must be negligible in $|x|$, the
adversary is given $f(x)$ and $1^n$ as an input, and hence must run polynomially
only in $|f(x)| + n$. This is a common technique called \emph{padding}, wherein
algorithms are given an extra input of $1^n$ to ensure they have enough time to
run.

We do not know that one-way functions exist. In fact, while the existence of
one-way functions implies that P $\neq$ NP, the converse is not
known\footnote{\cite{impagliazzo-1995} gives a classic discussion of the
  implications of various resolutions to P vs. NP on cryptography, including the
case where P $\neq$ NP but one-way functions nevertheless do not exist.}.
However, as in the following examples, we have excellent candidates under fairly
modest assumptions.

\begin{ex}[Factoring {\cite[subsection 2.3]{pass-shelat-2020}}]
  Suppose that for any adversary $\cA$ and for uniform random choice of $x = pq$
  for primes $p$ and $q$, \[
    \Pr[\cA(x) = \{p, q\}] = \negl[\max\{|p|, |q|\}].
  \] This is the \emph{factoring hardness assumption}, for which there is
  substantial evidence. Then $(x,y)\mapsto xy$ is one-way.
\end{ex}

\begin{ex}[Discrete Logarithm {\cite[subsection 8.3.2]{katz-lindell-2014}}]
  Let $G$ be any fixed group. The \emph{discrete logarithm hardness assumption}
  for $G$ is that, for any adversary $\cA$ and for uniform random choice of
  $g\in G$ and $h\in\<g\>$ such that $h = g^k$, \[
    \Pr[\cA(g,h) = k] = \negl[|g|].
  \]
  Under the discrete logarithm hardness assumption, $(g,k)\mapsto g^k$ is one-way.

  The discrete logarithm hardness assumption is known to be false for certain groups, such
  as the additive groups $\ZZ_p$ for prime $p$, in which case $g^k = gk$ and the
  Euclidean algorithm solves the problem. However, it is believed to hold for
  groups such as $\ZZ^*_p$ for sufficiently big prime $p$. For a survey of
  various versions of this assumption, see~\cite{sadeghi-steinerr-2002}.
\end{ex}

\subsection{Proofs by Reduction}

Many cryptographic definitions, including \Cref{def:one-way function},
take the form \emph{for any adversary $\cA$, natural number $n$, and uniform
random choice of input $x$ such that $|x| = n$, some predicate on the output
of $\cA$ has negligible probability.} The basic technique for proving results
using these definitions is called \emph{proof by reduction}. The idea is to
reduce one problem into another by starting with an arbitrary adversary
attacking the second and showing construct an adversary attacking the first,
such that the probability of their successes is related. If we assume the first
problem is hard, then by studying the structure of the reduction we can learn
about the hardness of the second problem. As such, we often say that reductions
prove \emph{relative hardness results}, so that for instance \Cref{ex:reduction}
below proves the hardness of $g$ relative to $f$.


More specifically, to prove hardness of a problem $\Pi$ relative to $\Pi'$, a
proof by reduction generally goes as follows:
\begin{enumerate}
  \item Fix an arbitrary adversary $\cA$ attacking a problem $\Pi$.
  \item Construct an adversary $\cA'$ attacking a problem $\Pi'$ which:
    \begin{enumerate}
      \item Receives an input $x'$ to $\Pi'$.
      \item Translates $x'$ into an input $x$ to $\Pi$.
      \item Simulates $\cA(x)$, getting back an output $y$ which solves
        $\Pi(x)$.
      \item Translates $y$ into an output $y'$ which solve $\Pi(x')$.
    \end{enumerate}
  \item Analyze the structure of the translations to conclude that $\cA'$ solves
    $\Pi'$ with probability related to that with which $\cA$ solves $\Pi$.
  \item Given the hardness assumptions on $\Pi'$, conclude relative hardness of
    $\Pi$.
\end{enumerate}

The point is that $\cA'$'s job is to ``simulate'' the problem $\Pi$ to $\cA$,
using the data it gets from $\Pi'$ to construct an input to $\Pi$. We illustrate
this concept now.

\begin{ex}[a straightforward proof by reduction {\cite[subsection
  2.4.1]{pass-shelat-2020}}]\label{ex:reduction} Let $f$ be a one-way function. Then we claim $g:
  (x,y)\mapsto (f(x), f(y))$ is a one-way function. We can compute $g$ in
  polynomial time by computing $f$ twice, so it remains to show that $g$ is hard
  to invert.

  Let $\cA$ be any adversary. We will construct an adversary $\cA'$ such
  that, if $\cA$ can non-negligibly invert $g$, then $\cA'$ can non-negligibly
  invert $f$.

  The adversary $\cA'$ takes input $1^n$ and $y$. It then uniformly randomly
  chooses $u$ of length $n$ and computes $v = f(u)$, which is possible because $f$ is easy to
  compute. Now $\cA'$ computes $(u', x') := \cA(1^{2n}, (v,y))$ and outputs
  $x'$.

  When $\cA'$ simulates $\cA$, it passes $v$, which is $f(u)$ for a uniform
  random $u$, and $y$, which is (on well-formed inputs) $f(x)$ for a uniform
  random $x$. Thus, this looks like exactly the input that $\cA$ would
  ``expect'' to receive if it is attempting to break $g$. As such, whenever
  $\cA$ successfully inverts $g$, $\cA'$ successfully inverts $f$. Since
  everything is uniform we may pass to probabilities, and so: \begin{align*}
    \Pr &[g(\cA(1^{2n}, g(u, x))) = g(u, x)] \\
        &= \Pr[g(\cA(1^{2n}, (f(u), f(x)))) = (f(u), f(x))] &&\text{by definition of $g$}  \\
        &\leq \Pr[f(\cA'(1^n, f(x))) = f(x)] &&\text{by the above argument} \\
        &= \negl &&\text{by the hardness assumption for $f$}.
     \end{align*}

   Thus $g$ is one-way.
\end{ex}

Comparing this example to the above schema, we see that the problem $\Pi'$ is to
invert $f$, while the problem $\Pi$ is to invert $g$. The input $x'$ to $\Pi'$
is $y$, while the computed input $x$ to $\Pi$ is $(v, y)$. The output $y$ of
$\cA$ is $(x', u')$, while the computed output $y'$ is $x'$.

Diagramatically, we can represent the algorithm $\cA'$ as follows:
\[
  \begin{pic}
    \setlength\minimummorphismwidth{6mm}
    \node[morphism] (A) at (0,0) {$\cA$};
    \draw ([xshift=-2.5pt]A.south west) to ++(0,-1.5) node[left] {$y$};
    \draw ([xshift=-2.5pt]A.north west) to ++(0,1.05) node[left] {$x'$};
		\draw ([xshift=2.5pt]A.south east) to ++(0,-.5) node[state,scale=0.75] {\normalsize$\$$};
    \draw ([xshift=2.5pt]A.north east) to ++(0,.5) node[right] {};
    \draw[dotted] (-.8, -1.5) rectangle (.9, 1);
    \node at (1.2, -1.3) {$\cA'$\punctuation{.}};
  \end{pic}
\]

While this is not standard notation in cryptography, it will be useful for our future purposes.
We read these diagrams---called \emph{circuit} or \emph{string diagrams}---from
bottom to top. This diagram says that $\cA'$ is an algorithm which takes $y$,
uniformly randomly generates another input (this is what the $\$$ means), calls
$\cA$, and returns its first output.

\subsection{Computational Indistinguishability}

Computational indistinguishability formalizes the notion of two probability
distributions which ``look the same'' to adversarial processes. We begin with
probability distributions, but because we want to do asymptotic analysis, we
will eventually need to switch to working with sequences of probability
distributions.

\begin{dfn}[computational advantage]\label{def:computational advantage}
  Let $X$ and $Y$ be probability distributions. The \emph{computational
  advantage} of an adversary $\cD$, called the
  \emph{distinguisher}, over $X$ and $Y$ is \[
    \ca_\cD(X, Y) = \left|\Pr_{x\from X}[\cD(x) = 1] - \Pr_{y\from Y}[\cD(y) = 1]\right|.
  \]
\end{dfn}

The idea is that the distinguisher $\cD$ is trying to guess whether its input
was drawn from $X$ or $Y$; the computational advantage is how often it can do
so.

\begin{prop}\label{thm:advantage is metric}
  Let $\cD$ be a fixed distinguisher. Then $\ca_\cD$ is a pseudometric on the
  space of probability distributions over an underlying set $A$.
\end{prop}

\begin{proof}
  Symmetry and non-negativity are immediate from the definition. To show the
  triangle inequality, let $X$, $Y$, and $Z$ be probability distributions over
  $A$. Let \[
    \hat{x} = \Pr_{x\from X}[\cD(x) = 1],
    \] and similarly for $\hat{y}$ and $\hat{z}$. Then, \[
  \ca_\cD(X, Z) = \left|\hat{x} - \hat{z}\right| \leq \left|\hat{x} -
  \hat{y}\right| + \left|\hat{y} - \hat{z}\right| = \ca_\cD(X, Y) +
  \ca_\cD(Y, Z).\qedhere
\]
\end{proof}

We now turn to the asymptotic case.

\begin{dfn}[probability ensemble]\label{def:probability ensemble}
  A \emph{probability ensemble} is a sequence $\{X_n\}$ of probability
  distributions.
\end{dfn}

We say that two ensembles are computationally indistinguishable if there is no
efficient way to tell between them. Formally:

\begin{dfn}[computational indistinguishability]\label{def:computational indistinguishability}
  Two probability ensembles $\{X_n\}$ and $\{Y_n\}$ are \emph{computationally
  indistinguishable} if for any (non-uniform PPT) distinguisher $\cD$ and any
  natural number $n$,
  \[
    \ca_\cD(X_n, Y_n) = \negl.
  \]
  In this case, we write $\{X_n\}\cind\{Y_n\}$.
\end{dfn}

\begin{rmk}
  A natural thought is to define a metric on probability distributions by
  $\ca(X, Y) = \sup_{\cD}\ca_\cD(X, Y)$, and extend to ensembles by asking
  that $\ca(X_n, Y_n) = \negl$. Unfortunately, this does not quite yield the
  correct notion, as there exist ensembles which are computationally
  indistinguishable, but have sequences of distinguishers whose advantages for
  any fixed $n$ converge to $1$.
\end{rmk}

\begin{prop}
  Computational indistinguishability is an equivalence relation on the space of
  probability ensembles over a fixed set $A$.
\end{prop}

\begin{proof}
  Reflexivity and symmetry follow from the case of distributions. To show
  transitivity, let $\{X_n\} \cind \{Y_n\}$ and $\{Y_n\} \cind \{Z_n\}$. Let
  $\cD$ be any distinguisher. Then for any $n$, \begin{align*}
    \ca_\cD(X_n, Z_n) &\leq \ca_\cD(X_n, Y_n) + \ca_\cD(Y_n, Z_n) &&\text{by the triangle inequality} \\
                        &= \negl + \negl &&\text{by assumption} \\
                        &= \negl. &&\qedhere
  \end{align*}
\end{proof}

It is necessary to be precise about what is being claimed here. Transitivity
states that for any \emph{constant, finite sequence} of probability ensembles,
if each is computationally indistinguishable from its neighbors, then the two
ends of the sequence are computationally indistinguishable. In cryptography, we
sometimes want to consider the more general case of a countable sequence of
probability ensembles. We can do slightly better than the previous result:

\begin{prop}
  Let $\{X^k\}$ be a sequence of probability ensembles, so
  that each $X^k = \{X^k_n\}$ is itself a sequence of probability distributions.
  Let $\{X^i\}\cind\{X^{i+1}\}$ for each $i$. Let $\{Y_n = X^{K(n)}_n\}$ for
  some polynomial $K$. Then $\{X^1_n\}\cind\{Y_n\}$.
\end{prop}

\begin{proof}
  Let $\cD$ be any distinguisher. Then for any $n$, \begin{align*}
    \ca_\cD(X^1_n, Y_n) &= \ca_\cD(X^1_n, X^{K(n)}_n) \\
                          &\leq \ca_\cD(X^1_n, X^2_n) + \cdots + \ca_\cD(X^{K(n) - 1}_n, X^{K(n)}_n) \\
                          &= K(n)\negl \\
                          &= \negl.
   \end{align*}
   In particular, the last equality follows because $K$ is polynomial.
\end{proof}

On the other hand, the result does not hold for arbitrary $K$. As we will see,
this is a fundamental limitation for cryptographic composition: we only expect
composition to work up to polynomial bounds.

One more closure result is valuable:

\begin{prop}
  Let $\{X_n\}\cind\{Y_n\}$, and let $\cM$ be a non-uniform PPT algorithm. Then
  $\{\cM(X_n)\} \cind \{\cM(Y_n)\}$.
\end{prop}

\begin{proof}
  The proof is by reduction. Let $\cD$ be a distinguisher. Then construct $\cD'$
  which, on input $x$, simulates $\cD(\cM(x))$. Then $\cD'$ outputs $1$ on $x$ if and
  only if $\cD$ outputs $1$ on $\cM(x)$, so \[
    \ca_\cD(\cM(X_n), \cM(Y_n)) = \ca_{\cD'}(X_n, Y_n) = \negl
  \] by the computational indistinguishability assumption.
\end{proof}

\subsection{Interactive and Zero-Knowledge Computation}
\label{sec:interactive computation}
\label{sec:zero-knowledge}

Cryptographic protocols do not occur in a vacuum; instead, they rely on
computations involving multiple parties. We formalize such situations using the
notion of interactive computation. In general, a model of interaction depends on
the underlying model of computation; this is for instance the case with the
popular notion of interactive Turing machines~\cite[Definition
4.2.1]{goldreich-2001}. As our approach in this chapter has been
model-independent, we can only give an informal discussion of interaction.

An \emph{interactive computation} consists of a finite number of \emph{parties},
which we think of as algorithms $\cA_i$, who may potentially communicate by
sending messages to each other, and whose behavior may change in response to
messages they receive. There may be limitations on these messages: for instance,
it may only be possible to send messages broadcasted to all the parties, or it
may be possible to send ``private'' party-to-party messages. When necessary, we
will always be clear about our assumptions here. An \emph{interactive protocol}
just consists of descriptions of some interactive algorithms $\<\cA_1, \dots,
\cA_n\>$.

At the start of an interactive computation, there is a \emph{global input} $x$
known to all parties, and each party $\cA_i$ may have a \emph{private} or
\emph{auxiliary input} $x_i$ known only to itself. At the end of the
computation, each party may make some output, the sequence of which we denote
$\<\cA_1, \dots, \cA_n\>(x, x_1, \dots, x_n)$, so that party $i$'s output is
$\<\cA_1, \dots, \cA_n\>(x, x_1, \dots, x_n)_i$. When any of these algorithms
are potentially probabilistic, we think of this value as a distribution over
possible outputs, and we always assume that the internal randomness of the
parties is independent.

The \emph{view} of a party is roughly all of the information it has available to
it over the course of the computation. This includes the global input, its
private input, any random bits it uses, and all the messages it receives. We
denote the view of party $i$ by $\view_i^{\<\cA_1, \dots, \cA_n\>}(x, x_1,
\dots, x_n)$. When the algorithms are clear from context, we may omit the
superscript. Importantly, while each private input $x_k$ is a parameter of each
view $\view_i$, the view does not necessarily include each of these inputs; they
are parameters merely because they may affect the messages received by party
$i$.

The \emph{running time} of an interactive algorithm $\cA$ is now the function
$T_\cA: \NN\to\NN$ which, for any $n$, gives the maximum number of ``steps'' it
takes $\cA$ to halt over any choice of:
\begin{itemize}
  \item global input $x$ and private input $y$ of total length $n = |x| + |y|$;
  \item other algorithms involved in the computation;
  \item internal randomness of either $\cA$ or any of the other algorithms
    involved in the computation.
\end{itemize}
Essentially, when we say an algorithm is polynomial-time, we mean it is
\emph{always} polynomial-time, no matter what. We sometimes assume that each
algorithm has a ``clock'' that it uses to count the number of steps it has taken
and ensure it halts in some fixed polynomial number of steps.

We often think of interactive computations as being indexed by a \emph{security
parameter} $n\in\NN$. The idea is that instead of asking each algorithm to be
polynomial in its inputs, we ask them to be polynomial in $n$, with the
stipulation that the inputs themselves are no more than polynomial in $n$, so
that each algorithm has time to read its own inputs. Intuitively, the security
parameter represents a ``tuning'' of the security of the system, so that higher
security incurs greater computational cost but gives stronger security
guarantees. Regardless, this notion can be incorporated into the above model by
padding the global input with the string $1^n$, as was done
in ~\Cref{def:one-way function}.

This model of interactive computation allows us to formalize the idea of a party
``learning something'' from an interaction. We say that an interactive
protocol $\<\cA_1, \dots, \cA_n\>$ is \emph{zero-knowledge for party $i$} if
there exists a non-uniform PPT algorithm $\cS$ such that
for any choice of inputs $(x, x_1, \dots, x_n)$, \[
  \cS(x, x_i) \cind \view_i^{\<\cA_1, \dots, \cA_n\>}(x, x_1, \dots, x_n).
\]
The idea is that the ``simulator'' $\cS$ gets only the inputs to $\cA_i$ and is
responsible for producing a distribution that is indistinguishable from the
actual view of $\cA_i$. If they can do this, then $\cA_i$ must not have learned
anything that they could not have computed directly from their inputs.

More often, we want to consider the situation where $\cA_i$ is supposed to learn
\emph{something} from the computation, but should not learn anything
\emph{extra}.

\begin{dfn}[Zero-Knowledge]\label{def:zero-knowledge}
  Let $f$ be a function. An interactive protocol $\<\cA_1, \dots, \cA_n\>$ is
  \emph{zero-knowledge for party $i$ relative to $f$} if there exists a non-uniform
  PPT $\cS$ such that for any choice of
  inputs $(x, x_1, \dots, x_n)$, \[
  \cS(x, x_i, f(x, x_1, \dots, x_n)) \cind \view_i^{\<\cA_1, \dots, \cA_n\>}(x, x_1, \dots, x_n).
\]
\end{dfn}

Sometimes we may also want to view $f$ as a function of the internal randomness
of each of the algorithms in the protocol, but we will not need to be so careful
here.

One more distinction is important. In the above definition, we are asking that
the simulator produces a distribution which is negligibly close, in the sense of
computational indistinguishability, to the actual view. While this is all that
is possible in many situations in practice, we could ask for the stronger
condition that the produced distribution is \emph{identical} to the view. We
call this notion \emph{perfect} or \emph{information-theoretic} zero-knowledge,
and refer to \Cref{def:zero-knowledge} as
\emph{computational} zero-knowledge when we wish to emphasize the distinction.

\section{Examples}

\subsection{Encryption}

A lot of the machinery defined in the previous section was originally developed
in the 1970s and 80s for the purpose of analyzing \emph{encryption problems},
culminating in the development of a \emph{public-key encryption protocol}
by~\cite{goldwasser-micali-1982}. The idea of an encryption problem is that a
party Alice has a message $m$ in the \emph{message space} $\cM_n$ which they want
to send to Bob, but any message they send to Bob must also be sent to the
eavesdropping Eve. In the simpler \emph{private-key encryption problem}, Alice
and Bob share some secret key $k$ from the \emph{key space $\cK_n$}, which is
unknown to Eve.

In this case, the problem reduces to a choice of three probabilistic
polynomial-time algorithms\footnote{Here, as is general practice, we omit the
dependence of the message and key spaces on the security parameter $n$.}:\begin{itemize}
  \item $\alg{Gen}$, which takes as input a security parameter $1^n$ and
    outputs a key $k\in\cK$;
  \item $\alg{Enc}$, which takes as input a security parameter $1^n$, a key
    $k\in\cK$, and a message $m\in\cM$, and outputs a ciphertext $c\in\cM$;
  \item $\alg{Dec}$, which takes as input a security parameter $1^n$, a key
    $k\in\cK$, and a ciphertext $c\in\cM$, and outputs a message $m\in\cM$.
\end{itemize}

An encryption scheme is \emph{correct} if for any choice of $m$ and $k$ output
by $\alg{Gen}(1^n)$, \[
  \alg{Dec}(1^n, k, \alg{Enc}(1^n, k, m)) = m.
\]

It is not hard to show that we may assume that $\alg{Gen}$ outputs a key chosen
uniformly at random from $\cK$. In this case, and now using the language
of~\Cref{sec:interactive computation}, we
say that a private-key encryption scheme is an interactive protocol consisting
of three interactive algorithms $\cA$, $\cB$, and $\cE$, where:
\begin{itemize}
  \item the global input is $1^n$, the security parameter;
  \item $\cA$ gets a uniform random key $k\in\cK$ and a message $m\in\cM$ as private input;
  \item $\cB$ gets the same uniform random key $k$ as private input;
  \item $\cE$ gets no private input;
  \item $\cA$ and $\cB$ may only send messages to each other if they also send
    the message to $\cE$.
\end{itemize}

We then say that an encryption scheme is \emph{correct} if $\cB$ outputs $m$ at
the end of the protocol. We say that an encryption scheme is \emph{secure} if it
is zero-knowledge for $\cE$; explicitly, if there exists a non-uniform PPT $\cS$
such that for any choice of security parameter $n$ and message $m$, \[
  \cS(1^n) \cind \view_\cE^{\<\cA, \cB, \cE\>}(1^n, k, m),
\]where the randomness of the second distribution is over both the randomness of
the algorithms and uniform random choice of $k$.

The point is that the eavesdropper should learn nothing from the interaction,
while the intended recipient should learn the message. Note that we may convert
the previous formulation to this one by having $\cA$ compute $c = \alg{Enc}(1^n,
k, m)$, send it to both other parties, and have $\cB$ output $\alg{Dec}(1^n,
m)$.

A correct-but-not-secure encryption scheme is easy to construct: simply have
$\cA$ send $m$ to $\cB$ as a message, and $\cB$ output that message. To show
that this is not secure, for any simulator $\cS$ we must give a choice of $1^n$
and $m$ and a distinguisher $\cD$ which distinguishes $\cS(1^n)$ from
$\view_\cE(1^n, k, m) = \{1^n, m\}$, where $k$ is chosen uniformly at random.
Note that we may choose the message $m$, and hence hard-code it into our
distinguisher, but the key $k$ must be chosen uniformly at random, because it is
not a proper input to the protocol. Regardless, we can simply choose any $m$ and
have $\cD$ output $1$ if and only if its input is $m$; since $\cS$ must be fixed
before the choice of $m$, this will distinguish between the distributions.

We can also construct a secure-but-not-correct encryption scheme, in which all
three simply do nothing. In this case, the view of $\cE$ is just the global
input $1_n$, which is exactly the input given to $\cS$. We can thus let $\cS$
compute the identity, in which case the two distributions are identical and so
indistinguishable.

We now give a secure and correct private key encryption scheme, called the
\emph{one-time pad}. Let $\{G_n\}$ be a sequence of additive finite groups\footnote{
  We also want that $\{G_n\}$ is \emph{efficiently sampleable}, so that it is
  possible to generate an element from it uniformly at random in polynomial
  time.
} such that $|G_n| = \Omega(2^n)$, for instance $G_n = \ZZ_2^n$. We let $\cM_n =
\cK_n = G_n$. Given a message $m$ and key $k$, $\cA$ computes $c = m + k$, which
it sends to $\cB$ (and $\cE$). $\cB$ then computes $c - k$, which it outputs.

Correctness of this scheme is immediate, as $\cB$ outputs $c - k = m + k - k =
m$. To prove security, our goal is to construct a simulator $\cS$ such that
$\cS(1^n)$ is indistinguishable from $\view_\cE = \{1^n, m + k\}$. Because
addition by $m$ is a bijection, and $k$ is chosen uniformly at random, the
distribution $\{m + k\}$ is just a uniform random sample from $G_n$. As such, we
simply let $\cS(1^n)$ draw $g$ uniformly at random from $G_n$ and output $\{1^n,
g\}$. This is again a perfectly-secure encryption scheme, since the two
distributions are identical.

% \subsection{Functionalities}

% An \emph{$n$-party functionality} is a function $f:
% \cM_1\times\dots\times\cM_n\to \cM'_1\times\dots\times\cM'_n$.

% \section{Composition}
